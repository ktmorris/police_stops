% Options for packages loaded elsewhere
\PassOptionsToPackage{unicode}{hyperref}
\PassOptionsToPackage{hyphens}{url}
%
\documentclass[
  12pt,
]{article}
\usepackage{amsmath,amssymb}
\usepackage{lmodern}
\usepackage{iftex}
\ifPDFTeX
  \usepackage[T1]{fontenc}
  \usepackage[utf8]{inputenc}
  \usepackage{textcomp} % provide euro and other symbols
\else % if luatex or xetex
  \usepackage{unicode-math}
  \defaultfontfeatures{Scale=MatchLowercase}
  \defaultfontfeatures[\rmfamily]{Ligatures=TeX,Scale=1}
\fi
% Use upquote if available, for straight quotes in verbatim environments
\IfFileExists{upquote.sty}{\usepackage{upquote}}{}
\IfFileExists{microtype.sty}{% use microtype if available
  \usepackage[]{microtype}
  \UseMicrotypeSet[protrusion]{basicmath} % disable protrusion for tt fonts
}{}
\makeatletter
\@ifundefined{KOMAClassName}{% if non-KOMA class
  \IfFileExists{parskip.sty}{%
    \usepackage{parskip}
  }{% else
    \setlength{\parindent}{0pt}
    \setlength{\parskip}{6pt plus 2pt minus 1pt}}
}{% if KOMA class
  \KOMAoptions{parskip=half}}
\makeatother
\usepackage{xcolor}
\IfFileExists{xurl.sty}{\usepackage{xurl}}{} % add URL line breaks if available
\IfFileExists{bookmark.sty}{\usepackage{bookmark}}{\usepackage{hyperref}}
\hypersetup{
  pdftitle={Supplementary Information},
  hidelinks,
  pdfcreator={LaTeX via pandoc}}
\urlstyle{same} % disable monospaced font for URLs
\usepackage[margin=1in]{geometry}
\usepackage{longtable,booktabs,array}
\usepackage{calc} % for calculating minipage widths
% Correct order of tables after \paragraph or \subparagraph
\usepackage{etoolbox}
\makeatletter
\patchcmd\longtable{\par}{\if@noskipsec\mbox{}\fi\par}{}{}
\makeatother
% Allow footnotes in longtable head/foot
\IfFileExists{footnotehyper.sty}{\usepackage{footnotehyper}}{\usepackage{footnote}}
\makesavenoteenv{longtable}
\usepackage{graphicx}
\makeatletter
\def\maxwidth{\ifdim\Gin@nat@width>\linewidth\linewidth\else\Gin@nat@width\fi}
\def\maxheight{\ifdim\Gin@nat@height>\textheight\textheight\else\Gin@nat@height\fi}
\makeatother
% Scale images if necessary, so that they will not overflow the page
% margins by default, and it is still possible to overwrite the defaults
% using explicit options in \includegraphics[width, height, ...]{}
\setkeys{Gin}{width=\maxwidth,height=\maxheight,keepaspectratio}
% Set default figure placement to htbp
\makeatletter
\def\fps@figure{htbp}
\makeatother
\setlength{\emergencystretch}{3em} % prevent overfull lines
\providecommand{\tightlist}{%
  \setlength{\itemsep}{0pt}\setlength{\parskip}{0pt}}
\setcounter{secnumdepth}{5}
\newlength{\cslhangindent}
\setlength{\cslhangindent}{1.5em}
\newlength{\csllabelwidth}
\setlength{\csllabelwidth}{3em}
\newlength{\cslentryspacingunit} % times entry-spacing
\setlength{\cslentryspacingunit}{\parskip}
\newenvironment{CSLReferences}[2] % #1 hanging-ident, #2 entry spacing
 {% don't indent paragraphs
  \setlength{\parindent}{0pt}
  % turn on hanging indent if param 1 is 1
  \ifodd #1
  \let\oldpar\par
  \def\par{\hangindent=\cslhangindent\oldpar}
  \fi
  % set entry spacing
  \setlength{\parskip}{#2\cslentryspacingunit}
 }%
 {}
\usepackage{calc}
\newcommand{\CSLBlock}[1]{#1\hfill\break}
\newcommand{\CSLLeftMargin}[1]{\parbox[t]{\csllabelwidth}{#1}}
\newcommand{\CSLRightInline}[1]{\parbox[t]{\linewidth - \csllabelwidth}{#1}\break}
\newcommand{\CSLIndent}[1]{\hspace{\cslhangindent}#1}
\usepackage{rotating}
\usepackage{setspace}
\usepackage{booktabs}
\usepackage{longtable}
\usepackage{array}
\usepackage{multirow}
\usepackage{wrapfig}
\usepackage{float}
\usepackage{colortbl}
\usepackage{pdflscape}
\usepackage{tabu}
\usepackage{threeparttable}
\usepackage{threeparttablex}
\usepackage[normalem]{ulem}
\usepackage{makecell}
\usepackage{xcolor}
\ifLuaTeX
  \usepackage{selnolig}  % disable illegal ligatures
\fi

\title{Supplementary Information}
\author{}
\date{\vspace{-2.5em}}

\begin{document}
\maketitle

{
\setcounter{tocdepth}{2}
\tableofcontents
}
\pagenumbering{gobble}
\pagebreak

\doublespacing

\hypertarget{administrative-matching-robustness-check}{%
\section{Administrative Matching Robustness Check}\label{administrative-matching-robustness-check}}

Our models exploring the turnout effects of traffic stops in Hillsborough County, Florida, require that we merge administrative records using the identifiers in the data. This runs the risk of identifying false positives. To test the prevalence of false positives in our administrative matching procedure, we use the test developed by Meredith and Morse (\protect\hyperlink{ref-Meredith2014}{2014}). By systematically permuting the birth dates in one set of records, we can see whether false positive matches are a major concern. In Table 1 we begin by merging all names and dates of birth in the traffic stop data with the names and dates of birth in the Hillsborough County registered voter file. We then add and subtract 35 days from the birth dates in the traffic stop data. If there are no false positives, these records should match with no records from the registered voter file.

\begin{singlespace}
\begin{table}[H]

\caption{\label{tab:shift-dobs-chunk}\label{tab:change-dobs} Results of Shifting Birthdates}
\centering
\begin{tabular}[t]{cc}
\toprule
Group & \makecell[c]{Number of Matches Between\\Traffic Stop and Voter File Records}\\
\midrule
Actual Birthdate & 263,149\\
Birthdate + 35 Days & 78\\
Birthdate - 35 Days & 60\\
\bottomrule
\end{tabular}
\end{table}
\end{singlespace}

As the table makes clear, more than a quarter-million registered voters in Hillsborough County match at least one record in the traffic stop database when merging by first and last name, and date of birth. Once we permute the birth dates, however, the match rate drops dramatically---to 60 or 78, depending on how these dates of birth are permuted. This translates into a false positive rate of roughly 0.03 percent. We consider this rate of false positives too low to meaningfully impact our results.

\hypertarget{full-regression-tables-for-administrative-test}{%
\section{Full Regression Tables for Administrative Test}\label{full-regression-tables-for-administrative-test}}

In the body of this manuscript, we present only the overall treatment effects for the police stops in Hillsborough County, which are effectively averaged across all three years. Here, in Tables 2--4, we present the results for each group of treated and control voters. In Table 2, all treated voters were stopped between the 2012 and 2014 elections, and all controls were stopped between the 2014 and 2016 elections. In Table 3, treated voters were stopped between 2014 and 2016, while controls were stopped between 2016 and 2018. Finally, Table 4 presents the treatment effect for voters stopped between 2016 and 2018, relative to their controls stopped between the 2018 and 2020 elections. In every year, there is a statistically significant, negative treatment effect for non-Black voters. In 2014 and 2016, the effect is significantly smaller for Black individuals, though in 2018 the treatment effect for Black and non-Black voters is statistically indistinguishable.

In Table 5, we present the full regression table for the overall models, with all covariates included.

\begin{singlespace}
\input{"../temp/dind_reg_y2014-11-04.tex"}
\end{singlespace}

\begin{singlespace}
\input{"../temp/dind_reg_y2016-11-08.tex"}
\end{singlespace}

\begin{singlespace}
\input{"../temp/dind_reg_y2018-11-06.tex"}
\end{singlespace}

\begin{singlespace}
\input{"../temp/dind_reg_yoverall.tex"}
\end{singlespace}

\hypertarget{regression-table-for-survey-data}{%
\section{Regression Table for Survey Data}\label{regression-table-for-survey-data}}

Here we present the regression table reported in the national survey data section of the manuscript. In model 1 we test whether personal or proximal contact with a police stop is differentially associated with turnout for Black and non-Black respondents. In this model, \emph{Stopped in Past 12 Months} captures the relationship between police stops and non-Black respondents; \emph{Stopped in Past 12 Months × Black} tests whether this relationship is different for Black respondents.

Models 2 and 3 whether the relationship is different for other non-white groups. Finally, model 4 tests the relationship between turnout and a historical arrest.

\begin{singlespace}
\input{"../temp/anes_raw.tex"}
\end{singlespace}

Model 1 in Table 6 shows that Black individuals who had been stopped by the police (or had a family member stopped) in the preceding 12 months were 9.6 percentage points more likely to vote in 2020, other things equal; they were not related to turnout for non-Black respondents. Police stops are not, however, associated with different turnout effects for other non-white groups. Moreover, as discussed in the body of the paper, historical arrests were uniformly associated with a decrease in turnout of 3 percentage points for Black and non-Black respondents alike.

\newpage

\hypertarget{references}{%
\section*{References}\label{references}}
\addcontentsline{toc}{section}{References}

\hypertarget{refs}{}
\begin{CSLReferences}{1}{0}
\leavevmode\vadjust pre{\hypertarget{ref-Meredith2014}{}}%
Meredith, Marc, and Michael Morse. 2014. {``Do {Voting Rights Notification Laws Increase Ex-Felon Turnout}?''} \emph{The ANNALS of the American Academy of Political and Social Science} 651 (1): 220--49. \url{https://doi.org/10.1177/0002716213502931}.

\end{CSLReferences}

\end{document}
